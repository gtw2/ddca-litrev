\section{Other Uses of Dynamic Deployment}
The concept of dynamic demand-driven deployment is not new,
and has been used in various applications. Various algorithms
are used to optimize deployment of facility or utilization
of given sources.

To maximize fleet utilization and minimize
operating costs, airlines predict future demands
and optimize their flight schedules and aircraft
types using linear programming methods \cite{berge_demand_1993}.
Benefits and costs of ticket pricing, aircraft assignment, and crew scheduling are
evaluated with a linear optimization method called ``Demand Driven Dispatch''
\cite{shebalov_practical_2009}, a type of deterministic optimization method.

A sawmill facility that processes timber contains 
many steps, from sawing, drying to planning. An
agent-based simulation is used to analyse demand-
driven production planning for a manufacturing system in Eastern Canada \cite{yáñez_agent-based_2009}.
The facility unit can be represented by a group of many facilities,
for example, source, sawing operations, drying operations, warehouse,
make agent, and delivery. In the simulation, each agent is given its parameters
and model its procedures.

The demand agent communicates to the make agent of the demand plan,
which will cause the make agent to communicate to the source agent. Then the
analysed and planned supply chain will be moved back to the make agent and
to the deliver agent.

\begin{figure}
	\includegraphics{width=\linewidth}{timber_process.jpg}
	\caption{Agent Coordination Protocol}
	\label{fig:timber_process}
	\cite{yáñez_agent-based_2009}
\end{figure}
